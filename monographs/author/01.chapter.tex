%%%%%%%%%%%%%%%%%%%%% chapter.tex %%%%%%%%%%%%%%%%%%%%%%%%%%%%%%%%%
%
% sample chapter
%
% Use this file as a template for your own input.
%
%%%%%%%%%%%%%%%%%%%%%%%% Springer-Verlag %%%%%%%%%%%%%%%%%%%%%%%%%%
%\motto{And Ken Said Let Everything Be A File...And Then There Was Light...}
\chapter{THE REAL AND COMPLEX NUMBER SYSTEMS}
\label{intro} % Always give a unique label
% use \chaptermark{}
% to alter or adjust the chapter heading in the running head


% \abstract{
    % test
% }
\section{INTRODUCTION}
A satisfactory discussion of the main concepts of analysis (such as convergence, continuity, differentiation, and integration) 
must be based on an accurately defined number concept. We shall not, however, enter into any discussion of 
the axioms that govern the arithmetic of the integers, but assume familiarity with the rational numbers 
(i.e., the numbers of the form $\frac{m}{n}$, where $m$ and $n$ are integers and $n\not=0$).\\
The rational number system is inadequate for many purposes, both as a field and as an ordered set. (These terms will 
be defined in Secs. ~\ref{to be added} and ~\ref{to be added}.) For instance, there is no rational $p$ such that 
$p^2=2$. (We shall prove this presently.) This leads to the introduction of so-called ``irrational numbers'' 
which are often written as infinite decimal expansions and are considered to be ``approximated'' by the corresponding 
finite decimals. Thus the sequence
\begin{equation*}
    1, 1.4, 1.41, 1.414, 1.4142, \dots
\end{equation*}
``tends to $\sqrt{2}$.'' But unless the irrational number $\sqrt{2}$ has been clearly defined, the question must arise: 
Just what is it that this sequence ``tends to''?\\
\indent This sort of question can be answered as soon as the so-called ``real number system'' is constructed.

\subsection*{\textbf{Example}}
\label{subsec:1}
We now show that the equation
\begin{equation}
    \label{eq:01}
    p^2=2
\end{equation}
is not satisfied by any rational $p$. If there were such a $p$, we could write $p=\frac{m}{n}$ where $m$ and $n$ 
are integers that are not both even. Let us assume this is done. Then ~\ref{eq:01} implies
\begin{equation}
    \label{eq:02}
    m^2=2n^2,
\end{equation}
This shows that $m^2$ is even. Hence $m$ is even (if $m$ were odd, $m^2$ would be odd), and so $m^2$ is divisible by $4$.
It follows that the right side of ~\ref{eq:02} is divisible by $4$, so that $n^2$ is even, which implies that $n$ 
is even.\\
\indent The assumption that ~\ref{eq:01} holds thus leads to the conclusion that both $m$ and $n$ are even, contrary 
to our choice of $m$ and $n$. Hence ~\ref{eq:01} is impossible for the rational $p$.\\
\indent We now examine this situation a little more closely. Let $\mathbf{A}$ be the set of all positive rationals $p$ such that $p^2 < 2$ 
and let $B$ consist of all positive rationals $p$ such that $^2 > 2$. We shall show that $\mathbf{A}$ \textit{contains no largest number and}
 $\mathbf{B}$ \textit{contains no smallest}.\\
 \indent More explicitly, for every $p$ in $\mathbf{A}$ we can find a rational $p$ in $\mathbf{A}$ such that 
 $p<q$, and for every $p$ in $\mathbb{B}$ we can find a rational $q$ in $\mathbf{B}$ such that $q<p$.\\
 \indent To do this, we associate with each rational $p>0$ the number
 \begin{equation}
    \label{eq:03}
    q=p-\frac{p^2-2}{p+2}=\frac{2p+2}{p+2}
 \end{equation}
\noindent Then
\begin{equation}
    \label{eq:04}
    q^2-2=\frac{2(p^2-2)}{(p+2)^2}
\end{equation}
\indent If $p$ is in $\mathbf{A}$ then $p^2-2<0$, ~\ref{eq:03} shows that $q>p$, and ~\ref{eq:04} shows that
$q^2<2$. Thus $q$ is in $\mathbf{A}$.\\
\indent If $p$ is in $\mathbf{B}$ then $p^2-2>0$, ~\ref{eq:03} shows that $0<q<p$, and ~\ref{eq:04} shows that
$q^2>2$. Thus $q$ is in $\mathbf{B}$.



~\cite{ARTIN}

\begin{definition}
    A field is a set $F$ with two operations, called \textit{addition} and \textit{multiplication},
    which satisfy the following so-called ``field axioms'' ~\ref{AXIOMSA}, and ~\ref{AXIOMSM}, and ~\ref{AXIOMSD}:


\subsubsection*{\textbf{Axioms for addition}}
\label{AXIOMSA}
\begin{itemize}
    \item If $x \in F$ and $y \in F$, then their sum $x+y$ is in $F$.
    \item Addition is commutative: $x+y=y+x$ for all $x,y \in F$
    \item Addition is associative: $(x+y)+z=x+(y+z)$ for all $x,y,z \in F$.
    \item $F$ contains an element $0$ such that $0+x=x$ for every $x \in F$.
    \item To every $x \in F$ corresponds an element $-x \in F$ such That
        \begin{equation*}
            x+(-x)=0.
        \end{equation*} 
\end{itemize}
\subsubsection*{\textbf{Axioms for multiplication}}
\label{AXIOMSM}
\begin{itemize}
    \item If $x \in F$ and $y \in F$, then their product $xy$ is in $F$.
    \item Multiplication is commutative: $xy=yx$ for all $x,y \in F$.
    \item Multiplication is associative: $(xy)z=x(yz)$ for all $x,y,z \in F$.
    \item $F$ contains an element $1 \neq 0$ such that $1x=x$ for every $x \in F$.
    \item If $x \in F$ and $x \neq 0$ then there exists an element $1/x \in F$ such that
    \begin{equation*}
        x.(1/x)=1.
    \end{equation*}
\end{itemize}

\subsubsection*{\textbf{The distributive law}}
\label{AXIOMSD}

\begin{equation*}
    x(y+z)=xy+xz
\end{equation*}
holds for all $x,y,z \in F$.
\end{definition}


\begin{theorem}
Theorem IS BEST USED LIKE THIS
\end{theorem}

\subsection*{\textbf{Remark}}
\label{sec:2}

\subsection*{\textbf{Definition}}
\label{sec:3}

\subsection*{\textbf{Definition}}
\label{sec:4}

\section{ORDERED SETS}







\subsection*{\textbf{Example}}
\label{sec:5}
\subsection*{\textbf{Example}}
\label{sec:6}
\subsection*{\textbf{Example}}
\label{sec:7}
\subsection*{\textbf{Example}}
\label{sec:8}
\subsection*{\textbf{Example}}
\label{sec:9}
\subsection*{\textbf{Example}}
\label{sec:10}
\subsection*{\textbf{Example}}
\label{sec:11}


\section{FIELDS}

\subsection*{\textbf{Example}}
\label{sec:12}


\subsection*{\textbf{Example}}
\label{sec:14}
\subsection*{\textbf{Example}}
\label{sec:15}
\subsection*{\textbf{Example}}
\label{sec:16}
\subsection*{\textbf{Example}}
\label{sec:17}
\subsection*{\textbf{Example}}
\label{sec:18}


\section{THE REAL FIELD}

\subsection*{\textbf{Example}}
\label{sec:19}
\subsection*{\textbf{Example}}
\label{sec:20}
\subsection*{\textbf{Example}}
\label{sec:21}
\subsection*{\textbf{Example}}
\label{sec:22}


\section*{THE EXTENDED REAL NUMBER SYSTEM}
\subsection*{\textbf{Example}}
\label{sec:23}



\section{THE COMPLEX FIELD}

\subsection*{\textbf{Example}}
\label{sec:24}

\subsection*{\textbf{Example}}
\label{sec:25}


\subsection*{\textbf{Example}}
\label{sec:26}

\subsection*{\textbf{Example}}
\label{sec:27}
\subsection*{\textbf{Example}}
\label{sec:28}
\subsection*{\textbf{Example}}
\label{sec:29}
\subsection*{\textbf{Example}}
\label{sec:30}
\subsection*{\textbf{Example}}
\label{sec:31}
\subsection*{\textbf{Example}}
\label{sec:32}
\subsection*{\textbf{Example}}
\label{sec:33}
\subsection*{\textbf{Example}}
\label{sec:34}
\subsection*{\textbf{Example}}
\label{sec:35}

\section{EUCLIDEAN SPACES}
\subsection*{\textbf{Example}}
\label{sec:36}

\subsection*{\textbf{Example}}
\label{sec:37}
\subsection*{\textbf{Example}}
\label{sec:38}

\section{APPENDIX}

\subsection*{\textbf{Step}}
\label{sec:37}












\label{sec:0}
\begin{programcode}{basic usage}
\begin{verbatim}
brew install
    fgjn
\end{verbatim}
\end{programcode}



\section{Section Heading}
\label{sec:1}
Use the template \emph{chapter.tex} together with the document class SVMono (monograph-type books) or SVMult (edited books) to style the various elements of your chapter content conformable to the Springer Nature layout.

\section{Section Heading}
\label{sec:2}
% Always give a unique label
% and use \ref{<label>} for cross-references
% and \cite{<label>} for bibliographic references
% use \sectionmark{}
% to alter or adjust the section heading in the running head
Instead of simply listing headings of different levels we recommend to let every heading be followed by at least a short passage of text. Furtheron please use the \LaTeX\ automatism for all your cross-references and citations.

Please note that the first line of text that follows a heading is not indented, whereas the first lines of all subsequent paragraphs are.

\eject

Use the standard \verb|equation| environment to typeset your equations, e.g.
%
\begin{equation}
a \times b = c\;,
\end{equation}
%
however, for multiline equations we recommend to use the \verb|eqnarray| environment\footnote{In physics texts please activate the class option \texttt{vecphys} to depict your vectors in \textbf{\itshape boldface-italic} type - as is customary for a wide range of physical subjects.}.
\begin{eqnarray}
\left|\nabla U_{\alpha}^{\mu}(y)\right| &\le&\frac1{d-\alpha}\int
\left|\nabla\frac1{|\xi-y|^{d-\alpha}}\right|\,d\mu(\xi) =
\int \frac1{|\xi-y|^{d-\alpha+1}} \,d\mu(\xi)\qquad  \\
&=&(d-\alpha+1) \int\limits_{d(y)}^\infty
\frac{\mu(B(y,r))}{r^{d-\alpha+2}}\,dr \le (d-\alpha+1)
\int\limits_{d(y)}^\infty \frac{r^{d-\alpha}}{r^{d-\alpha+2}}\,dr
\label{eq:10}
\end{eqnarray}

\enlargethispage{24pt}

\subsection{Subsection Heading}
\label{subsec:2}
Instead of simply listing headings of different levels we recommend to let every heading be followed by at least a short passage of text. Further on please use the \LaTeX\ automatism for all your cross-references\index{cross-references} and citations\index{citations} as has already been described in Sect.~\ref{sec:2}.

\begin{quotation}
Please do not use quotation marks when quoting texts! Simply use the \verb|quotation| environment -- it will automatically be rendered in the preferred layout.
\end{quotation}


\subsubsection{Subsubsection Heading}
Instead of simply listing headings of different levels we recommend to let every heading be followed by at least a short passage of text. Furtheron please use the \LaTeX\ automatism for all your cross-references and citations as has already been described in Sect.~\ref{subsec:2}, see also Fig.~\ref{fig:1}\footnote{If you copy text passages, figures, or tables from other works, you must obtain \textit{permission} from the copyright holder (usually the original publisher). Please enclose the signed permission with the manucript. The sources\index{permission to print} must be acknowledged either in the captions, as footnotes or in a separate section of the book.}

Please note that the first line of text that follows a heading is not indented, whereas the first lines of all subsequent paragraphs are.

% For figures use
%
\begin{figure}[b]
\sidecaption
% Use the relevant command for your figure-insertion program
% to insert the figure file.
% For example, with the option graphics use
\includegraphics[scale=.65]{figure}
%
% If not, use
%\picplace{5cm}{2cm} % Give the correct figure height and width in cm
%
\caption{If the width of the figure is less than 7.8 cm use the \texttt{sidecapion} command to flush the caption on the left side of the page. If the figure is positioned at the top of the page, align the sidecaption with the top of the figure -- to achieve this you simply need to use the optional argument \texttt{[t]} with the \texttt{sidecaption} command}
\label{fig:1}       % Give a unique label
\end{figure}


\paragraph{Paragraph Heading} %
Instead of simply listing headings of different levels we recommend to let every heading be followed by at least a short passage of text. Furtheron please use the \LaTeX\ automatism for all your cross-references and citations as has already been described in Sect.~\ref{sec:2}.

Please note that the first line of text that follows a heading is not indented, whereas the first lines of all subsequent paragraphs are.

For typesetting numbered lists we recommend to use the \verb|enumerate| environment -- it will automatically render Springer's preferred layout.

\begin{enumerate}
\item{Livelihood and survival mobility are oftentimes coutcomes of uneven socioeconomic development.}
\begin{enumerate}
\item{Livelihood and survival mobility are oftentimes coutcomes of uneven socioeconomic development.}
\item{Livelihood and survival mobility are oftentimes coutcomes of uneven socioeconomic development.}
\end{enumerate}
\item{Livelihood and survival mobility are oftentimes coutcomes of uneven socioeconomic development.}
\end{enumerate}


\subparagraph{Subparagraph Heading} In order to avoid simply listing headings of different levels we recommend to let every heading be followed by at least a short passage of text. Use the \LaTeX\ automatism for all your cross-references and citations as has already been described in Sect.~\ref{sec:2}, see also Fig.~\ref{fig:2}.

Please note that the first line of text that follows a heading is not indented, whereas the first lines of all subsequent paragraphs are.

For unnumbered list we recommend to use the \verb|itemize| environment -- it will automatically render Springer's preferred layout.

\begin{itemize}
\item{Livelihood and survival mobility are oftentimes coutcomes of uneven socioeconomic development, cf. Table~\ref{tab:1}.}
\begin{itemize}
\item{Livelihood and survival mobility are oftentimes coutcomes of uneven socioeconomic development.}
\item{Livelihood and survival mobility are oftentimes coutcomes of uneven socioeconomic development.}
\end{itemize}
\item{Livelihood and survival mobility are oftentimes coutcomes of uneven socioeconomic development.}
\end{itemize}

\begin{figure}[t]
\sidecaption[t]
% Use the relevant command for your figure-insertion program
% to insert the figure file.
% For example, with the option graphics use
\includegraphics[scale=.65]{figure}
%
% If not, use
%\picplace{5cm}{2cm} % Give the correct figure height and width in cm
%
\caption{Please write your figure caption here}
\label{fig:2}       % Give a unique label
\end{figure}

\runinhead{Run-in Heading Boldface Version} Use the \LaTeX\ automatism for all your cross-references and citations as has already been described in Sect.~\ref{sec:2}.

\subruninhead{Run-in Heading Boldface and Italic Version} Use the \LaTeX\ automatism for all your cross-refer\-ences and citations as has already been described in Sect.~\ref{sec:2}\index{paragraph}.

\subsubruninhead{Run-in Heading Displayed Version} Use the \LaTeX\ automatism for all your cross-refer\-ences and citations as has already been described in Sect.~\ref{sec:2}\index{paragraph}.
% Use the \index{} command to code your index words
%
% For tables use
%
\begin{table}[!t]
\caption{Please write your table caption here}
\label{tab:1}       % Give a unique label
%
% For LaTeX tables use
%
\begin{tabular}{p{2cm}p{2.4cm}p{2cm}p{4.9cm}}
\hline\noalign{\smallskip}
Classes & Subclass & Length & Action Mechanism  \\
\noalign{\smallskip}\svhline\noalign{\smallskip}
Translation & mRNA$^a$  & 22 (19--25) & Translation repression, mRNA cleavage\\
Translation & mRNA cleavage & 21 & mRNA cleavage\\
Translation & mRNA  & 21--22 & mRNA cleavage\\
Translation & mRNA  & 24--26 & Histone and DNA Modification\\
\noalign{\smallskip}\hline\noalign{\smallskip}
\end{tabular}
$^a$ Table foot note (with superscript)
\end{table}
%
\section{Section Heading}
\label{sec:3}
% Always give a unique label
% and use \ref{<label>} for cross-references
% and \cite{<label>} for bibliographic references
% use \sectionmark{}
% to alter or adjust the section heading in the running head
Instead of simply listing headings of different levels we recommend to let every heading be followed by at least a short passage of text. Furtheron please use the \LaTeX\ automatism for all your cross-references and citations as has already been described in Sect.~\ref{sec:2}.

Please note that the first line of text that follows a heading is not indented, whereas the first lines of all subsequent paragraphs are.

If you want to list definitions or the like we recommend to use the Springer-enhanced \verb|description| environment -- it will automatically render Springer's preferred layout.

\begin{description}[Type 1]
\item[Type 1]{That addresses central themes pertainng to migration, health, and disease. In Sect.~\ref{sec:1}, Wilson discusses the role of human migration in infectious disease distributions and patterns.}
\item[Type 2]{That addresses central themes pertainng to migration, health, and disease. In Sect.~\ref{subsec:2}, Wilson discusses the role of human migration in infectious disease distributions and patterns.}
\end{description}

\subsection{Subsection Heading} %
In order to avoid simply listing headings of different levels we recommend to let every heading be followed by at least a short passage of text. Use the \LaTeX\ automatism for all your cross-references and citations citations as has already been described in Sect.~\ref{sec:2}.

Please note that the first line of text that follows a heading is not indented, whereas the first lines of all subsequent paragraphs are.

\begin{svgraybox}
If you want to emphasize complete paragraphs of texts we recommend to use the newly defined Springer class option \verb|graybox| and the newly defined environment \verb|svgraybox|. This will produce a 15 percent screened box 'behind' your text.

If you want to emphasize complete paragraphs of texts we recommend to use the newly defined Springer class option and environment \verb|svgraybox|. This will produce a 15 percent screened box 'behind' your text.
\end{svgraybox}


\subsubsection{Subsubsection Heading}
Instead of simply listing headings of different levels we recommend to let every heading be followed by at least a short passage of text. Furtheron please use the \LaTeX\ automatism for all your cross-references and citations as has already been described in Sect.~\ref{sec:2}.

Please note that the first line of text that follows a heading is not indented, whereas the first lines of all subsequent paragraphs are.

\begin{theorem}
Theorem text goes here.snwkeJFNKwjenfkwjenF
\end{theorem}
%
% or
%
\begin{definition}
Definition text goes here.
\end{definition}

\begin{proof}
%\smartqed
Proof text goes here.
%\qed
\end{proof}

\paragraph{Paragraph Heading} %
Instead of simply listing headings of different levels we recommend to let every heading be followed by at least a short passage of text. Furtheron please use the \LaTeX\ automatism for all your cross-references and citations as has already been described in Sect.~\ref{sec:2}.

Note that the first line of text that follows a heading is not indented, whereas the first lines of all subsequent paragraphs are.
%
% For built-in environments use
%
\begin{theorem}
Theorem text goes here.
\end{theorem}
%
\begin{definition}
Definition text goes here.
\end{definition}
%
\begin{proof}
%\smartqed
Proof text goes here.
%\qed
\end{proof}
%
%
\begin{trailer}{Trailer Head}
If you want to emphasize complete paragraphs of texts in an \verb|Trailer Head| we recommend to
use  \begin{verbatim}\begin{trailer}{Trailer Head}
...
\end{trailer}\end{verbatim}
\end{trailer}
%
\begin{question}{Questions}
If you want to emphasize complete paragraphs of texts in an \verb|Questions| we recommend to
use  \begin{verbatim}\begin{question}{Questions}
...
\end{question}\end{verbatim}
\end{question}
%
%
\begin{important}{Important}
If you want to emphasize complete paragraphs of texts in an \verb|Important| we recommend to
use  \begin{verbatim}\begin{important}{Important}
...
\end{important}\end{verbatim}
\end{important}
%
\clearpage
\begin{warning}{Attention}
If you want to emphasize complete paragraphs of texts in an \verb|Attention| we recommend to
use  \begin{verbatim}\begin{warning}{Attention}
...
\end{warning}\end{verbatim}
\end{warning}

\begin{programcode}{Program Code}
If you want to emphasize complete paragraphs of texts in an \verb|Program Code| we recommend to
use

\verb|\begin{programcode}{Program Code}|

\verb|\begin{verbatim}...\end{verbatim}|

\verb|\end{programcode}|

\end{programcode}
%
\begin{tips}{Tips}
If you want to emphasize complete paragraphs of texts in an \verb|Tips| we recommend to
use  \begin{verbatim}\begin{tips}{Tips}
...
\end{tips}\end{verbatim}
\end{tips}
%
%
\begin{overview}{Overview}
If you want to emphasize complete paragraphs of texts in an \verb|Overview| we recommend to
use  \begin{verbatim}\begin{overview}{Overview}
...
\end{overview}\end{verbatim}
\end{overview}
\clearpage
\begin{backgroundinformation}{Background Information}
If you want to emphasize complete paragraphs of texts in an \verb|Background|
\verb|Information| we recommend to
use

\verb|\begin{backgroundinformation}{Background Information}|

\verb|...|

\verb|\end{backgroundinformation}|
\end{backgroundinformation}
\begin{legaltext}{Legal Text}
If you want to emphasize complete paragraphs of texts in an \verb|Legal Text| we recommend to
use  \begin{verbatim}\begin{legaltext}{Legal Text}
...
\end{legaltext}\end{verbatim}
\end{legaltext}
%
\begin{acknowledgement}
If you want to include acknowledgments of assistance and the like at the end of an individual chapter please use the \verb|acknowledgement| environment -- it will automatically render Springer's preferred layout.
\end{acknowledgement}
%
\section*{Appendix}
\addcontentsline{toc}{section}{Appendix}
%
When placed at the end of a chapter or contribution (as opposed to at the end of the book), the numbering of tables, figures, and equations in the appendix section continues on from that in the main text. Hence please \textit{do not} use the \verb|appendix| command when writing an appendix at the end of your chapter or contribution. If there is only one the appendix is designated ``Appendix'', or ``Appendix 1'', or ``Appendix 2'', etc. if there is more than one.

\begin{equation}
a \times b = c
\end{equation}
% Problems or Exercises should be sorted chapterwise
\section*{Problems}
\addcontentsline{toc}{section}{Problems}
%
% Use the following environment.
% Don't forget to label each problem;
% the label is needed for the solutions' environment
\begin{prob}
\label{prob1}
If $r$ is rational $(r \neq 0)$ and $x$ is irrational, prove that $r+x$ and $rx$ are irrational.
\end{prob}

\begin{prob}
\label{prob2}
If $r$ is rational $(r\neq 0)$ and $x$ is irrational, prove that $r+x$ and $rx$ are irrational.
\end{prob}

\begin{prob}
\label{prob3}
If $r$ is rational $(r\not0)$ and $x$ is irrational, prove that $r+x$ and $rx$ are irrational.
\end{prob}

\begin{prob}
\label{prob4}
If $r$ is rational $(r\not0)$ and $x$ is irrational, prove that $r+x$ and $rx$ are irrational.
\end{prob}

\begin{prob}
\label{prob5}
If $r$ is rational $(r\not0)$ and $x$ is irrational, prove that $r+x$ and $rx$ are irrational.
\end{prob}



\begin{prob}
\label{prob6}
Fix $b > 1$.\\

\begin{enumerate}
    \item  If $m,n,p,q$ are integers, $n>0,q>0$, and $r=\frac{m}{n}=\frac{p}{q}$, prove that
    \begin{equation*}
        (b^m)^{1/n}=(b^p)^{1/q}
    \end{equation*}

    \item part 2
\end{enumerate}



\end{prob}

 

%%%%%%%%%%%%%%%%%%%%%%%%% referenc.tex %%%%%%%%%%%%%%%%%%%%%%%%%%%%%%
% sample references
% %
% Use this file as a template for your own input.
%
%%%%%%%%%%%%%%%%%%%%%%%% Springer-Verlag %%%%%%%%%%%%%%%%%%%%%%%%%%
%
% BibTeX users please use
% \bibliographystyle{}
% \bibliography{}
%
\Extrachap{FURTHER READING}

\biblstarthook{}


\begin{thebibliography}{99.}%


\bibitem{ARTIN} ARTIN, E.: ``The Gamma Function,'' Holt, Rinehart and Winston, Inc., New York, 1964.
\bibitem{BOAS} BOAS, R. P.: ``A Primer of Real Functions,'' Carus Mathematical Monograph No. 13, John Wiley \& Sons, Inc., New York, 1960.
\bibitem{BUCKANALYSIS} BUCK, R. C. (ed.): ``Studies in Modern Analysis,'' Prentice-Hall, Inc., Englewood Cliffs, N.J., 1962.
\bibitem{BUCKCALCULUS} BUCK, R. C. (ed.): ``Advanced Calculus'' 2d ed., McGraw-Hill Book Company, New York, 1965.
\bibitem{BURKILL} BURKILL, J. C.: ``The Lebesgue Integral,'' Cambridge University Press, New York, 1951.
\bibitem{DIEUDONNE} DIEUDONNE, J.: ``Foundations of Modern Analysis,'' Academic Press, Inc., New York, 1960.
\bibitem{FLEMING} FLEMING, w. H.: ``Functions of Several Variables,'' Addison-Wesley Publishing Company, Inc., Reading, Mass., 1965.
\bibitem{GRAVES} GRAVES, L. M.: ``The Theory of Functions of Real Variables,'' 2d ed., McGraw-Hill Book Company, New York, 1956.
\bibitem{HALMOSMEASURE} HALMOS, P. R.: ``Measure Theory,'' D. Van Nostrand Company, Inc., Princeton, N.J., 1950.
\bibitem{HALMOSVECTOR} HALMOS, P. R.: ``Finite-dimensional Vector Spaces,'' 2d ed., D. Van Nostrand Company, Inc., Princeton, N.J., 1958.
\bibitem{HARDYPURE} HARDY, G. H.: ``Pure Mathematics,'' 9th ed., Cambridge University Press, New York, 1947.
\bibitem{HARDYFOURIER} HARDY, G. H. and ROGOSINSKI, W.: ``Fourier Series,'' 2d ed., Cambridge University Press, New York, 1950.
\bibitem{HERSTEIN} HERSTEIN, I. N.: ``Topics in Algebra,'' Blaisdell Publishing Company, New York, 1964.
\bibitem{HEWITT} HEWITT, E., and STROMBERG, K. : ``Real and Abstract Analysis,'' Springer Publishing Co., Inc., New York, 1965.
\bibitem{KELLOGG} KELLOGG, O. D.: ``Foundations of Potential Theory,'' Frederick Ungar Publishing Co., New York, 1940.
\bibitem{KNOPP} KNOPP, K.: ``Theory and Application of Infinite Series,'' Blackie \& Son, Ltd., Glasgow, 1928.
\bibitem{LANDAU} LANDAU, E. G. H.: ``Foundations of Analysis,'' Chelsea Publishing Company, New York, 1951.
\bibitem{MCSHANE} MCSHANE, E. J.: ``Integration,'' Princeton University Press, Princeton, N.J., 1944.
\bibitem{NIVEN} NIVEN, I. M.: ``Irrational Numbers,'' Carus Mathematical Monograph No. 11, John Wiley \& Sons, Inc., New York, 1956.
\bibitem{ROYDEN} ROYDEN, H. L.: ``Real Analysis,'' The Macmillan Company, New York, 1963.
\bibitem{RUDIN} RUDIN, W.: ``Real and Complex Analysis,'' 2d ed., McGraw-Hill Book Company, New York, 1974.
\bibitem{SIMMONS} SIMMONS, G. F.: ``Topology and Modern Analysis,'' McGraw-Hill Book Company, New York, 1963.
\bibitem{SINGER} SINGER, 1. M., and THORPE, J. A.: ``Lecture Notes on Elementary Topology and Geometry,'' Scott, Foresman and Company, Glenview, Ill., 1967.
\bibitem{SMITH} SMITH, K. T.: ``Primer of Modern Analysis'' Bogden and Quigley, Tarrytown-on-Hudson, N.Y., 1971.
\bibitem{SPIVAK} SPIVAK, M.: ``Calculus on Manifolds,'' W. A. Benjamin, Inc., New York, 1965.
\bibitem{THURSTON} THURSTON, H. A.: ``The Number System,'' Blackie \& Son, Ltd., London-Glasgow, 1956.




\end{thebibliography}

